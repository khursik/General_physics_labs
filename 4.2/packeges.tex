\NeedsTeXFormat{LaTeX2e}
\ProvidesClass{kiarticle}
\ProcessOptions

\LoadClass
    [12pt, a4paper, oneside, final]{article}


%%%%%%%%%%%%%%%%%%%%%%%%%%%%%%%%%%%%%%%%%%%%%%%%%%%%%%%%%%%%%%%%%%%%%%%%%%%%%%%
%                                 Математика                                  %
%%%%%%%%%%%%%%%%%%%%%%%%%%%%%%%%%%%%%%%%%%%%%%%%%%%%%%%%%%%%%%%%%%%%%%%%%%%%%%%

\RequirePackage{mathtext}
\RequirePackage{mathtools}
\RequirePackage{amsmath}
\RequirePackage{amsfonts}
\RequirePackage{amssymb}
\RequirePackage[all,warning]
               {onlyamsmath}
    \AtBeginDocument{\catcode`\$=3}
\RequirePackage{cancel}
\RequirePackage{icomma}
\RequirePackage{mathrsfs}
\RequirePackage{bigints}
\RequirePackage{mathrsfs}
%\RequirePackage{physics}
\RequirePackage{indentfirst}
\mathtoolsset{showonlyrefs=true} % Показывать номера только у тех формул, на которые есть \eqref{} в тексте.


\RequirePackage{udef}
\RequirePackage{iunits}
%%%%%%%%%%%%%%%%%%%%%%%%%%%%%%%%%%%%%%%%%%%%%%%%%%%%%%%%%%%%%%%%%%%%%%%%%%%%%%%
%                              Формат документа                               %
%%%%%%%%%%%%%%%%%%%%%%%%%%%%%%%%%%%%%%%%%%%%%%%%%%%%%%%%%%%%%%%%%%%%%%%%%%%%%%%

\RequirePackage[T2A,OT1]           {fontenc}
\RequirePackage[utf8]              {inputenc}
\RequirePackage[english, russian]  {babel}
\RequirePackage[ left     = 1.5cm,
                 right    = 1.5cm,
                 top      = 2.0cm,
                 bottom   = 1.25cm,
                 includefoot,
                 footskip = 1.25cm ]{geometry}
\setlength    {\parskip}        { .5em plus .15em minus .08em }
%\setlength    {\parindent}      { .0em }
\renewcommand {\baselinestretch}{ 1.07 }


%%%%%%%%%%%%%%%%%%%%%%%%%%%%%%%%%%%%%%%%%%%%%%%%%%%%%%%%%%%%%%%%%%%%%%%%%%%%%%%
%                                  Графика                                    %
%%%%%%%%%%%%%%%%%%%%%%%%%%%%%%%%%%%%%%%%%%%%%%%%%%%%%%%%%%%%%%%%%%%%%%%%%%%%%%%

\RequirePackage{graphicx}
\RequirePackage{float}
\RequirePackage{wrapfig}
\RequirePackage{color}
\RequirePackage{tikz}
\RequirePackage{pgfplots}
    \pgfplotsset{compat=newest}


%%%%%%%%%%%%%%%%%%%%%%%%%%%%%%%%%%%%%%%%%%%%%%%%%%%%%%%%%%%%%%%%%%%%%%%%%%%%%%%
%                        Нумерация, заголовки и подписи                       %
%%%%%%%%%%%%%%%%%%%%%%%%%%%%%%%%%%%%%%%%%%%%%%%%%%%%%%%%%%%%%%%%%%%%%%%%%%%%%%%

\RequirePackage{fancyref}
\RequirePackage{fancyhdr}

\RequirePackage{secdot}
\RequirePackage{caption}


%%%%%%%%%%%%%%%%%%%%%%%%%%%%%%%%%%%%%%%%%%%%%%%%%%%%%%%%%%%%%%%%%%%%%%%%%%%%%%%
%                              Таблицы и списки                               %
%%%%%%%%%%%%%%%%%%%%%%%%%%%%%%%%%%%%%%%%%%%%%%%%%%%%%%%%%%%%%%%%%%%%%%%%%%%%%%%

\RequirePackage{enumitem}
    \setlist{leftmargin =   1cm,
             topsep     =  .0\parskip,
             itemsep    = -.0\parskip,
             resume
            }
\RequirePackage{booktabs}
\RequirePackage{array}
    \renewcommand{\arraystretch}{ 1.20 }


    \ProvidesPackage{udef}

    %%%%%%%%%%%%%%%%%%%%%%%%%%%%%%%%%%%%%%%%%%%%%%%%%%%%%%%%%%%%%%%%%%%%%%%%%%%%%%%
    %                                 Определения                                 %
    %%%%%%%%%%%%%%%%%%%%%%%%%%%%%%%%%%%%%%%%%%%%%%%%%%%%%%%%%%%%%%%%%%%%%%%%%%%%%%%
    
    \newcommand{\tw}{\textwidth}
    \newcommand{\lw}{\linewidth}
    \newcommand{\bs}{\baselineskip}
    \newcommand{\HRule}{\rule{\linewidth}{0.5mm}}
    
    %\newcommand{\dd}[2]{\frac{d #1}{d #2}}
    \newcommand{\pdd}[2]{\frac{\partial #1}{\partial #2}}
    \newcommand{\ddc}[3]{\left( \frac{\partial #1}{\partial #2} \right)_{\! #3}}
    %\newcommand{\abs}[1]{\left| #1 \right|}
    \newcommand{\divc} {\mathop{\raisebox{-2pt}{\vdots}}}
    
    \renewcommand{\iff}     {\quad \Longleftrightarrow \quad}
    \newcommand  {\then}    {\quad \Longrightarrow \quad}
    \newcommand  {\subst}[1]{\begin{Vmatrix} #1 \end{Vmatrix}}
    \newcommand  {\qed}     {\hfill\ensuremath{\square}}
    
    \newcommand  {\const}{\ensuremath{\mathrm{const}}}
    \renewcommand{\Re}   {\operatorname{Re}}
    \renewcommand{\Im}   {\operatorname{Im}}
    \DeclareMathOperator {\Ker}{Ker}
    \DeclareMathOperator {\Rg} {Rg}
    \DeclareMathOperator {\Int}{int}
    
    %%
    \renewcommand{\epsilon}{\ensuremath{\varepsilon}}
    \renewcommand{\phi}{\ensuremath{\varphi}}
    \renewcommand{\kappa}{\ensuremath{\varkappa}}
    %\renewcommand{\le}{\ensuremath{\leqslant}}
    %\renewcommand{\leq}{\ensuremath{\leqslant}}
    %\renewcommand{\ge}{\ensuremath{\geqslant}}
    %\renewcommand{\geq}{\ensuremath{\geqslant}}
    \renewcommand{\emptyset}{\varnothing}
    
    %%
    
    %DeclareMathOperator{\sgn}{\mathop{sgn}}
    \newcommand{\te}{\ensuremath{\; \Rightarrow \;}}
    \newcommand{\y}{\ensuremath{\angle}}
    \newcommand{\ABC}{\ensuremath{\triangle ABC\,}}
    %\newcommand{\tr}{\ensuremath{\triangle}}
    \newcommand{\ca}{\ensuremath{\cos\alpha}}
    \newcommand{\sa}{\ensuremath{\sin\alpha}}
    \newcommand{\cb}{\ensuremath{\cos\beta}}
    \newcommand{\sib}{\ensuremath{\sin\beta}}
    \newcommand{\ov}{\ensuremath{\overline}}
    \newcommand{\x}{\cdot}
    \newcommand{\de}{\varDelta}
    \newcommand{\st}{\ensuremath{\longrightarrow}}
    %%
    \newcommand{\al}{\ensuremath{\alpha}}
    \newenvironment{solv}{ \begin{center}  \fbox }{ \end{center}}
    \DeclareMathOperator {\di}{div}
    \DeclareMathOperator {\rot}{rot}
    \DeclareMathOperator {\inv}{inv}
    \DeclareMathOperator {\diag}{diag}
    %Перенос знаков в формулах (по Львовскому)
    \newcommand*{\hm}[1]{#1\nobreak\discretionary{}	{\hbox{$\mathsurround=0pt #1$}}{}}
    \newcommand{\ii}{\ensuremath{\mathbf{i}}}
    \newcommand{\jj}{\ensuremath{\mathbf{j}}}
    \newcommand{\kk}{\ensuremath{\mathbf{k}}}
    \newcommand{\vv}{\ensuremath{|v\rangle}}
    \newcommand{\lm}{\ensuremath{\Lambda}}
    \newcommand{\gm}{\ensuremath{\Gamma_{\mu\nu}}}
    \newcommand{\ekv}{\ensuremath{\Leftrightarrow}}
    \newcommand{\dpa}{\ensuremath{\partial}}
    %%
    \newcommand{\s}[1]{\left( #1 \right)}
    \newcommand{\sak}[1]{\left\{ #1 \right\}}
    \newcommand{\sab}[1]{\left[ #1 \right]}
    \newcommand{\sys}[1]{
        \left\{
        \begin{aligned} #1 \end{aligned}
        \right.
    }
    \newcommand{\del}{\ensuremath{\delta}}
    \newcommand{\co}{\ensuremath{\mathrm{const}}}
    \newcommand{\D}{\ensuremath{\mathcal{D}}}
    \newcommand{\spdd}[2]{\left( \frac{\partial #1}{\partial #2} \right)}
    \newcommand{\Dd}[4]{\frac{\partial \left( #1, #2\right) }{\partial \left( #3, #4\right)}}
    
    \newcommand
    {\un}[1]
    {\ensuremath{\text{#1}}}
    \newcommand{\eds}{\ensuremath{ \mathscr{E}}}
    \newcommand{\ga}{\ensuremath{\gamma}}
    \newcommand{\an}{\ensuremath{\mathring{A}}}
    
    %%%%%%%%%%%%%%%%%%%%%%%%%%%%%%%%%%%%%%%%%%%%%%%%%%%%%%%%%%%%%%%%%%%%%%%%%%%%%%%
    %                                Патчи окружений                              %
    %%%%%%%%%%%%%%%%%%%%%%%%%%%%%%%%%%%%%%%%%%%%%%%%%%%%%%%%%%%%%%%%%%%%%%%%%%%%%%%
    
    % Cases with bracket
    \makeatletter
    \newenvironment{sqcases}{%
      \matrix@check\sqcases\env@sqcases
    }{%
      \endarray\right.%
    }
    \def\env@sqcases{%
      \let\@ifnextchar\new@ifnextchar
      \left\lbrack
      \def\arraystretch{1.2}%
      \array{@{}l@{\quad}l@{}}%
    }
    \makeatother
    
    
    % Auxiliary alignment and separators inside matrix environment
    \makeatletter
    \renewcommand*\env@matrix[1][*\c@MaxMatrixCols c]{%
      \hskip -\arraycolsep
      \let\@ifnextchar\new@ifnextchar
      \array{#1}}
    \makeatother
    
    %Ответ